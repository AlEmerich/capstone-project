\documentclass{article}
\usepackage[english]{babel}
\usepackage[square,comma,sort,numbers]{natbib}
\usepackage{glossaries}
\usepackage{hyperref}
\usepackage{graphicx}
\usepackage{xcolor}
\usepackage{amsmath}
\usepackage{multicol}
\usepackage[T1]{fontenc}
\renewcommand{\familydefault}{\sfdefault}
\setlength{\columnsep}{1cm}
\graphicspath{ {images/} }

% \makeglossaries
% \newglossaryentry{ZMP}{
  name=Zero Moment Point,
  description={A definition}
}

\title{\textbf{Capstone Report}\\Arise and walk, by Deep Deterministic Policy Gradient algorithm}
\date{\today}
\author{Guitard Alan}


\begin{document}
\raggedright
\maketitle

\section{Definition of the problem}

\subsection{Project Overview}

With this project, I want to tackle an important task of robotics: the ability
of a humanoïd to walk. This is a really interesting problem since we, humans, learn
to walk without any feedback from another agent, just by a training process. The
parents doesn't teach anything to the kids, the kids just learns. I think this
is very interesting to build such model of artificial intelligence in order to
to understand our brain a little bit more. \\
The anthropologic reason is not the only reason that task is important, it is
also important to build autonomous walking robots able to progress on every
fields. There are several areas of actions: In war, such a robot will be able to
reach wounded soldier to give medical treatment or bring the soldier back to a
safe place. It will avoid a medic human to risk his life in order to do that. In
hospital, we will gain time by giving task to the walking robot so doctors and
nurses will have more time to give to the patient. In every day life, we will be
able to get near from the future described by movies like ``I-Robot''. Even if
it questions the place of the robot in human society, I think making this kind of
robot is still a goal to reach in order to send the robot in places human cannot
or with difficulty go, like to another planet.\\
I am talking about walking but a neural network able to make a robot walk sure
will be able to tackle another difficult task. The difficulty in those task is
their continuous observation spaces. Indeed, in order to make a robot walk, we
have two possibilities. The first one is to learn over pixels of the
environment. I didn't choose this one because I don't think a human learn to
walk from its sight but rather from its body. That why I chose to watch the
position of its joints. With pixels, we will able to train the humanoïd to watch
its steps but only after it is able to walk.

\subsection{Problem Statement}

The goal of this project, precisely, is to train a 3D avatar to walk forward on
a plane fields. I will use \cite{1606.01540} as interface to the environment
because it gives a simple and general way to use all kind of environment. For
3-dimensional avatar, it provides a binding for Mujoco library but since
this library is paid, I will use a free similar one called
Roboschool\footnote{\href{https://github.com/openai/roboschool}} which propose
few environments among the one I will use, RoboschoolHumanoid-v1.\\
I am planning to use Reinforcement Learning to tackle this task with the
Deep Deterministic Policy Gradient algorithm
(\citeauthor{journals/corr/LillicrapHPHETS15}), an actor-critic algorithm.

\paragraph{Actor-critic} In Reinforcement Learning, many algorithms uses an
action-value function. It is function which returns the expected rewards after
taken an action from a state following a policy. It is defined like this:
\begin{equation}
  R = \sum_{}
\end{equation}

\subsection{Metrics}

\section{Analysis}

\subsection{Data Exploration}

\subsubsection{Observation space}

This environment has an observation space of 44 float values in the range [-5,
5] which is a concatenation a three vectors described as follows:
\begin{itemize}
\item{\textbf{more}: It is a vector of 8 values defined as follows:
    \begin{itemize}
    \item{The distance between the last position of the body and the current one.}
    \item{The sinus of the angle to the target.}
    \item{The cosinus of the angle to the target.}
    \item{The three next values is the X, Y and Z values of the matrix multiplication between
        \begin{itemize}
        \item{\[\left(
                \begin{matrix}
                  \cos(-yaw) & -\sin(-yaw) & 0 \\
                  \sin(-yaw) & \cos(yaw) & 0 \\
                  0 & 0 & 1
                \end{matrix}\right)
            \]}
        \item{The speed vector of the body.}
        \end{itemize}}
    \item{The roll value of the body}
    \item{The pitch value of the body}
    \end{itemize}}
\item{\textbf{j}: This is the current relative position of the joint described earlier and their current speed. The position is in the even position, and the speed in the odds (34 values).}
\item{\textbf{feet\_contact}: Boolean values, 0 or 1, for left and right feet, indicating if the respective feet is touching the ground or not.}
\end{itemize}

\subsubsection{Action space}
The action space is a vector of 17 float values in the range [-1, 1]. Each
value corresponds to the joints of the avatar by this order
\href{https://github.com/openai/roboschool/blob/master/roboschool/mujoco_assets/humanoid_symmetric.xml}{XML}:
\begin{multicols}{2}
  \begin{itemize}
  \item{abdomen\_y}
  \item{abdomen\_z}
  \item{abdomen\_x}
  \item{right\_hip\_x}
  \item{right\_hip\_z}
  \item{right\_hip\_y}
  \item{right\_knee}
  \item{left\_hip\_x}
  \item{left\_hip\_z}
  \item{left\_hip\_y}
  \item{left\_knee}
  \item{right\_shoulder1}
  \item{right\_shoulder2}
  \item{right\_elbow}
  \item{left\_shoulder1}
  \item{left\_shoulder2}
  \item{left\_elbow}
  \end{itemize}
\end{multicols}
  At each step, these values are applied to all the joints of the body by the code
\begin{verbatim}
for n,j in enumerate(self.ordered_joints):
    j.set_motor_torque( self.power*j.power_coef \
                         *float(np.clip(a[n], -1, +1)) )
\end{verbatim}

in the \verb?apply_action? function in the class which extends the
\verb?gym.Env? class (\verb?RoboschoolMujocoXmlEnv?) to set the torque value
into the respective motor.

\subsection{Exploratory Visualization}

\subsection{Algorithm and Techniques}

\paragraph{Reward}
The reward is a sum of 5 computed values: \begin{itemize}
  \item{\textbf{alive}: -1 or +1 wether is on the ground or not}
  \item{\textbf{progress}: potential minus the old potential. The potential is defined by
    the speed multiplied by the distance to target point, to the negative.}
  \item{\textbf{electricity\_cost}: The amount of energy needed for the last action}
  \item{\textbf{joints\_at\_limit\_cost}: The amount of collision between joints of body
      during the last action}
  \item{\textbf{feet\_collsion\_cost}: The amount of feet collision taken during the last action}
  \end{itemize}

\subsection{Benchmark}

\section{Methodology}

\subsection{Data Preprocessing}

\subsection{Implementation}

\subsection{Refinement}

\section{Results}

\subsection{Model Evaluation and Validation}

\subsection{Justification}

\section{Conclusion}

\subsection{Free-Form Visualization}

\subsection{Reflection}

\subsection{Improvement}

\bibliography{biblio}
\bibliographystyle{plainnat}

\end{document}