\documentclass{article}
\usepackage[english]{babel}

\usepackage[square,comma,sort,numbers]{natbib}
\usepackage{glossaries}
\usepackage{hyperref}
\usepackage{graphicx}
\usepackage{xcolor}
\usepackage{amsmath}
\usepackage[document]{ragged2e}
\usepackage{amssymb}
\usepackage{listings}
\usepackage{multicol}
\usepackage[T1]{fontenc}
\usepackage[font=scriptsize]{caption}

\definecolor{codegreen}{rgb}{0,0.6,0}
\definecolor{codegray}{rgb}{0.5,0.5,0.5}
\definecolor{codepurple}{rgb}{0.58,0,0.82}
\definecolor{backcolour}{rgb}{0.95,0.95,0.92}

\lstdefinestyle{mystyle}{
    backgroundcolor=\color{backcolour},   
    commentstyle=\color{codegreen},
    keywordstyle=\color{magenta},
    numberstyle=\tiny\color{codegray},
    stringstyle=\color{codepurple},
    basicstyle=\footnotesize,
    breakatwhitespace=false,         
    breaklines=true,                 
    captionpos=b,                    
    keepspaces=true,                 
    numbers=left,                    
    numbersep=5pt,                  
    showspaces=false,                
    showstringspaces=false,
    showtabs=false,                  
    tabsize=2
}
 
\lstset{style=mystyle}
 
\renewcommand{\familydefault}{\sfdefault}
\setlength{\columnsep}{1cm}
\graphicspath{ {images/} }


% \makeglossaries
% \newglossaryentry{ZMP}{
  name=Zero Moment Point,
  description={It specifies the point with respect to which dynamic reaction force at the contact of the foot with the ground does not produce any moment in the horizontal direction, i.e. the point where the total of horizontal inertia and gravity forces equals 0 (zero).}
}

\title{\textbf{Capstone Report}\\Arise and walk, by Deep Deterministic Policy Gradient algorithm}
\date{\today}
\author{Guitard Alan}


\begin{document}

\maketitle

\includegraphics[width=\textwidth]{roboschoolhumanoid}
\justify

\section{Definition of the problem}

\subsection{Project Overview}
\paragraph{}
With this project, I want to tackle an important task of robotics: the ability
of a humanoid to walk. This is a really interesting problem since we, humans, learn
to walk without any feedback from another agent, just by a training process. The
parents doesn't teach anything to the kids, the kids just learns. I think this
is very interesting to build such model of artificial intelligence in order to
to understand our brain a little bit more.
\paragraph{}
The anthropologic reason is not the only reason that task is important, it is
also important to build autonomous walking robots able to progress on every
fields. There are several areas of actions: In war, such a robot will be able to
reach wounded soldier to give medical treatment or bring the soldier back to a
safe place. It will avoid a medic human to risk his life in order to do that. In
hospital, we will gain time by giving task to the walking robot so doctors and
nurses will have more time to give to the patient. In every day life, we will be
able to get near from the future described by movies like ``I-Robot''. Even if
it questions the place of the robot in human society, I think making this kind of
robot is still a goal to reach in order to send the robot in places human cannot
or with difficulty go, like to another planet.
\paragraph{}
I am talking about walking but a neural network able to make a robot walk sure
will be able to tackle another difficult task. The difficulty in those task is
their continuous observation spaces. Indeed, in order to make a robot walk, we
have two possibilities. The first one is to learn over pixels of the
environment. I didn't choose this one because I don't think a human learn to
walk from its sight but rather from its body. That why I chose to watch the
position of its joints. With pixels, we will able to train the humanoid to watch
its steps but only after it is able to walk.

\subsection{Problem Statement}
\paragraph{}
The goal of this project, precisely, is to train a 3D avatar to walk forward on
a plane fields. I will use OpenAI Gym \cite{1606.01540} as an interface to the environment
because it gives a simple and general way to use all kind of environment. For
3-dimensional avatar, it provides a binding for Mujoco library but since
this library is paid, I will use a free similar one called
Roboschool\footnote{\href{https://github.com/openai/roboschool}
  {https://github.com/openai/roboschool}} which proposes
few environments among the one I will use, RoboschoolHumanoid-v1. I am planning
to use Reinforcement Learning to tackle this task with the Deep Deterministic
Policy Gradient algorithm (\citeauthor{journals/corr/LillicrapHPHETS15}), an
actor-critic algorithm. It is ``Deep'' because the actor and the critic is
designed with deep neural networks. This algorithm is mostly used when the
environment is a continuous space and since this environment never changes in
our case (it is a plane field without changes), we can use a deterministic policy. 

\paragraph{Actor Critic algorithm}

In Reinforcement Learning, many algorithms uses an action-value function.
It is a function which returns the value of an action $a$ from a state $s$
following a policy $\mu$. It is defined like this: 
  
\begin{equation}
  Q^\mu(s_t,a_t) = \mathbb{E}_{r_t,s_{t+1}\sim{E}}[r(s_t,a_t) + \gamma
  Q^\mu(s_{t+1}, \mu(s_{t+1}))]
\end{equation}

Besides that, the policy is the map of probabilities for an agent to go
to state $s_{t+1}$ from state $s$ by the action $a$:

\begin{equation}
  \mu(s,a) = P(a_t=a|s_t=s)
\end{equation}

The idea of actor-critic algorithms is to use the action-value function as a
critic which will give the $Q$ value of the action taken by the agent at each time
step. The higher the $Q$ is, the more the reward the agent will get by taking
that action in that state. The agent, here called the actor, will follow a
policy $\mu$ in order to chose the action and that policy will be updated by the
output of the critic. In Layman's terms, the actor is a child playing in the
sandbox and the critic is the parent watching him. When the child make an action
that may lead to a bad state, the parent gives him a bad reward in order to
change his behaviour (policy).

\subsection{Metrics}

\paragraph{}
I will need two set of metrics because I have two kind of session for the body:
it can walk and fail to stand or it can walk without falling. When the avatar
will fall, I will restart the session, because to teach it to stand up is
another kind of problem.
\paragraph{}
At the beggining of the training, the body will fall and fall again very quickly.
So my metrics during that period will be the amount of time step it took before
failing, the distance of the gravity center from the floor, the reward per
action and the global average reward. Since the actor and critic are neural
networks, I will also plot the loss of the critic network and their weights and
biases, to check if it is changing over times. With these informations, I will
be able to tell if my models is training well or if I have to adjust my parameters.
When the body will start to have less fall, some of these metrics will not be
informative anymore. I have to find metrics to evaluate the gait of the walk.
For that, I will plot the angle of the current body position from the start
position in respect to the axe the avatar will try to follow. 

\section{Analysis}

\subsection{Data Exploration: RoboschoolHumanoid-v1}

\subsubsection{Observation space}
\label{subsub:obs_space}

\paragraph{Definition} An observation space, or state space, is the shape and
the possible values an environment state could have. If the state is not
continuous, we can calculate the state space by counting all possible values.
For example, in a tic-tac-toe game, every square have 3 states so the state
space is $3^9 = 19,683$. Here, our state is continuous, meaning that our possible
values stands in a range. We can then just describe values and specify the range.

\paragraph{}
Our environment has an observation space of 44 float values in the range [-5,
5] which is a concatenation a three vectors described as follows:
\begin{itemize}
\item{\textbf{more}: It is a vector of 8 values defined as follows:
    \begin{itemize}
    \item{The distance between the last position of the body and the current one.}
    \item{The sinus of the angle to the target.}
    \item{The cosinus of the angle to the target.}
    \item{The three next values is the X, Y and Z values of the matrix multiplication between
        \begin{itemize}
        \item{\[\left(
                \begin{matrix}
                  \cos(-yaw) & -\sin(-yaw) & 0 \\
                  \sin(-yaw) & \cos(yaw) & 0 \\
                  0 & 0 & 1
                \end{matrix}\right)
            \]}
        \item{The speed vector of the body.}
        \end{itemize}}
    \item{The roll value of the body}
    \item{The pitch value of the body}
    \end{itemize}}
\item{\textbf{j}: This is the current relative position of the joint described earlier and their current speed. The position is in the even position, and the speed in the odds (34 values).}
\item{\textbf{feet\_contact}: Boolean values, 0 or 1, for left and right feet, indicating if the respective feet is touching the ground or not.}
\end{itemize}

\subsubsection{Action space}

\paragraph{Definition} An action space is like observation space but for action
an actor can take. In the tic-tac-toe example, the actor has 9 possible actions,
which is playing in one of the square. 
\paragraph{}
The action space is a vector of 17 float values in the range [-1, 1]. Each
value corresponds to the joints of the avatar by this order
\href{https://github.com/openai/roboschool/blob/master/roboschool/mujoco_assets/humanoid_symmetric.xml}{XML}:
\begin{multicols}{2}
  \begin{itemize}
  \item{abdomen\_y}
  \item{abdomen\_z}
  \item{abdomen\_x}
  \item{right\_hip\_x}
  \item{right\_hip\_z}
  \item{right\_hip\_y}
  \item{right\_knee}
  \item{left\_hip\_x}
  \item{left\_hip\_z}
  \item{left\_hip\_y}
  \item{left\_knee}
  \item{right\_shoulder1}
  \item{right\_shoulder2}
  \item{right\_elbow}
  \item{left\_shoulder1}
  \item{left\_shoulder2}
  \item{left\_elbow}
  \end{itemize}
\end{multicols}
  At each step, these values are applied to all the joints of the body by the code
\begin{lstlisting}[language=Python]
for n,j in enumerate(self.ordered_joints):
    j.set_motor_torque( self.power*j.power_coef \
                         *float(np.clip(a[n], -1, +1)) )
\end{lstlisting}

in the \verb?apply_action? function in the class which extends the
\verb?gym.Env? class (\verb?RoboschoolMujocoXmlEnv?) to set the torque value
into the respective motor.

\subsubsection{Reward}

\paragraph{Definition} A reward is a value given the information if the action
was good or not given the state. The definition of the reward function is a
critical aspect of reinforcement learning since it is the one who gives the most
valuable information during the training.
\paragraph{}
The reward is a sum of 5 computed values: \begin{itemize}
  \item{\textbf{alive}: -1 or +1 wether is on the ground or not}
  \item{\textbf{progress}: potential minus the old potential. The potential is defined by
    the speed multiplied by the distance to target point, to the negative.}
  \item{\textbf{electricity\_cost}: The amount of energy needed for the last action}
  \item{\textbf{joints\_at\_limit\_cost}: The amount of collision between joints of body
      during the last action}
  \item{\textbf{feet\_collsion\_cost}: The amount of feet collision taken during the last action}
  \end{itemize}

\subsection{Exploratory Visualization}

To visualize the exploration, we can simply use the render function of the gym
environment which uses \verb?Roboschool? library to create the 3D plane and
humanoid such as show in the figure \ref{fig:roboschoolhumanoid}.  

\begin{figure}[ht]
  \centering
  \includegraphics[width=\textwidth]{roboschoolhumanoid}
  \caption{Roboschool environment}
  \label{fig:roboschoolhumanoid}
\end{figure}

\subsection{Algorithm and Techniques}

\paragraph{Deep Deterministic Policy Gradient}

In DDPG, instead of compute manually the $Q$ value and the $\mu$, we
use neural networks, one for the actor and one for the critic. The figure
\ref{fig:actor-critic} shows how such a model can train.

\begin{figure}[ht]
  \centering
  \includegraphics[width=.5\textwidth]{actor-critic}
  \caption{First, the environment gives the first state to the actor and it chose the
action following the policy. The action is given back to the environment and a
reward is computed, meaning how good the action was for that state, and sent to
the critic network. The critic computes the $Q$ value of that action by
minimizing the loss and gives the error to the actor in order to let him train
on it.}
  \label{fig:actor-critic}
\end{figure}

The Temporal Difference learning (TD Error in the figure) is a machine learning
technique to, basically, let a network able to train with a little guess of the
future. In our case of walking, the critic should not only see the next state
but also estimate the chance the actor has to fall because of the action it
took. To do that, we will use target actor and target critic network. It is, at
the beggining, a copy of the actor and the critic but we will use these networks
to compute the target action for a state and $Q_{t+1}$ and, at each time steps,
we will update their weights $\theta^{Q'}$ with the actual network weights
$\theta^Q$ by the following formula called ``soft update'':  

\begin{equation}
\theta^{Q'} \leftarrow \tau\theta^Q + (1 - \tau)\theta^Q
\end{equation}

Not using target networks will cause divergence and none of the model will learn
a good policy or a good $Q$-value function.
\paragraph{}
We need to also design a Replay buffer to store transition $(s_t, a, r,
s_{t+1})$ and train over minibatch of random transition. In that way, we break
temporal correlations between transition and minimize variance between our
possible bad predictions.
\paragraph{}
A last thing we need to introduce is exploration noise. We want to add some noise to the
action the actor take during training in order to let it explore its world with
some randomness. A good way to compute noise is the Ornstein–Uhlenbeck process,
which is also known as Brownian motion. 

\begin{figure}[ht]
  \centering
  \includegraphics[width=\textwidth]{algoDDPG}
  \caption{Algorithm from \citeauthor{journals/corr/LillicrapHPHETS15} paper}
  \label{fig:algoDDPG}
\end{figure}

The figure \ref{fig:algoDDPG} describes the algorithm in a pseudo-code. The
reader can see that the actor doesn't learn the classical way, i.e. by
minimizing its loss. Instead of that, its weights is updated with the gradient
of the critic network after it trained. 

\subsection{Benchmark}

The most basic benchmark we have to test is the random benchmark. What will
happen in a worse case scenario ? 

\begin{figure}[ht]
  \centering
  \includegraphics[width=\textwidth]{randompolicy}
  \caption{Random Policy}
  \label{fig:randompolicy}
\end{figure}

The figure \ref{fig:randompolicy} shows that the agent fall and fall again,
without any chance to stand. The relevant information here is the average
reward. The average reward of the random policy over 500 epochs is around 20 and
the agent never lives more than 30 epochs.
\paragraph{}
Now we have a low benchmark, we have to dig into previous work of the community
in order to find the state-of-the-art paper for that task. The work of
\citeauthor{journals/corr/LillicrapHPHETS15} in their paper called
'Continuous control with deep reinforcement Learning' does not show results with
a humanoid environment but they did reach a good policy for 2d walker. Another
more recent studies\cite{GuLilGhaTurLev17} shows results of different algorithm
including DDPG. 

\begin{figure}[ht]
  \centering
  \includegraphics[width=\textwidth]{benchmark}
  \caption{Average return over episodes in HalfCheetah-v1 and Humanoid-v1 during
    learning, comparing Q-Prop against other model-free algorithms. Q-Prop with
    vanilla policy gradient outperforms TRPO on HalfCheetah. Q-Prop
    significantly outperforms TRPO in convergence time on Humanoid.}
  \label{fig:benchmark}
\end{figure}

The figure \ref{fig:benchmark} emphases that maybe DDPG is not a good choice to
such a complex environment like Humanoid because DDPG is really sensitive about
hyper-parameters changes. I still want to try to teach a robot to walk with that
algorithm by increasing the depth of the networks for example and see what
happened.  

\section{Methodology}

\subsection{Data Preprocessing}

The Roboschool environment is made for educational and research purpose. In that
matter, a lot of useful data preprocessing is already made. For the observation
vector, after building it as explained in section \ref{subsub:obs_space}, the
value is clipped between -5 and 5, meaning that if the environment provides a
value outside of these bounds, the value is set to the nearest bounds. This is
the same for action space but the value is clipped between -1 and 1. 

\subsection{Implementation}

\subsubsection{Replay buffer}

To implement that task, I started by writing the replay buffer. I created a
class \verb?Memory? which holds a queue as a member in order to store tuple of
$(s_t, a, r, s_{t+1})$. I defined the first method as follow:

\begin{lstlisting}[language=Python]
def remember(self, state, action, reward, next_state, done, state_range=None, action_range=None)
\end{lstlisting}

The \verb?done? value is a boolean telling if the state is terminal or not. The
state and the action range is a value to normalize the respective vector between
0 and 1. I think by this way neural networks will have better performance
because input is less sparsed. If not \verb?None?, these values must be equal to
the range between low and high value (2 for action and 10 for state in our
case).\\
The second needed method to define is the one to get samples from the queue randomly:

\begin{lstlisting}[language=Python]
def samples(self, batch_size)
\end{lstlisting}

We need to get random sample, not last one or older one, but random, in order to
break their correlations which would lead to decrease the performance of the model.

\subsubsection{Neural networks: Actor and Critic}

\subsubsection{Training Operations}

\subsubsection{Noise}

\subsubsection{Visualization}

\subsubsection{Hyper Parameters}

\subsection{Refinement}

\section{Results}

\subsection{Model Evaluation and Validation}

\subsection{Justification}

\section{Conclusion}

\subsection{Free-Form Visualization}

\subsection{Reflection}

\subsection{Improvement}

\bibliography{report}
\bibliographystyle{plainnat}

\end{document}