\documentclass{article}
\usepackage[english]{babel}
\usepackage[square,comma,sort,numbers]{natbib}
\usepackage{glossaries}
\usepackage{hyperref}
\usepackage{graphicx}
\usepackage{xcolor}
\usepackage{amsmath}
\usepackage{multicol}
\usepackage[T1]{fontenc}
\renewcommand{\familydefault}{\sfdefault}
\setlength{\columnsep}{1cm}
\graphicspath{ {images/} }

\makeglossaries
\newglossaryentry{ZMP}{
  name=Zero Moment Point,
  description={It specifies the point with respect to which dynamic reaction force at the contact of the foot with the ground does not produce any moment in the horizontal direction, i.e. the point where the total of horizontal inertia and gravity forces equals 0 (zero).}
}

\title{\textbf{Capstone Proposal}\\Arise and walk}
\date{\today}
\author{Guitard Alan}

\begin{document}
\raggedright
	\maketitle	
	\section{Domain Background}
  \paragraph{}
  For a human, the ability to walk is one of the most basic knowledge he learnt
  during his childhood. Since we are trying with AI to, when we can't do better,
  mimic the ability of human being, the idea of teaching a robot to walk comes
  naturally in our mind.
  \paragraph{}
  One of the first attempt to do that was to develop an algorithm by trying to
  understand the physics formula of that movement. In 1993,
  \citet{1993-TrunkMotion} tried to make a little robot by trunk motion
  using the ZMP (\gls{ZMP}) and the three
  axis (pitch, yaw and roll)\cite{1993-TrunkMotion}. In 1999, \citeauthor{1999-KHR-2}
  added the OGM (Optimal Gradient Method) to that approach
  to "optimizes the horizontal motion of a trunk to reduce the
  deviation of the calculated ZMP from its
  reference." \cite{1999-KHR-2} and design the KHR-2 robot.
  The result was good but the gait of the robot wasnot very natural because
  it is very difficult to take all the factors in account for making
  the robot walk with a human gait.
  \paragraph{}
  Nowadays, Boston Dynamics made a huge advances in humanoïd robotic by
  designing Atlas, who was able to do a perfect backflip, or walk on a non-flat
  ground. In terms of simulation, \citeauthor{2013-TOG-MuscleBasedBipeds}
  desgined a muscle based avatar with two legs which can take a lot of shape.
  According to that shape and environment (e.g. the gravity), they were able to
  teach the creature to stand and walk, and it learned the proper gait (one
  creature with small legs figured out itself it is easier to
  jump).\cite{2013-TOG-MuscleBasedBipeds}
  All that studies could be used in many areas. Healthcare institute could
  design better articial arm or leg, or could give more efficient companion
  robots to their patient. That last one could be use in all fields of life,
  like a buttler. In a more ludic ways, we will be able to design a more
  realistic gait for our 3D avatar, even for non-player characters. 
  
	\section{Problem Statement}

  The problem I want to solve is then to teach a 3D avatar to walk. That kind of
  problem is solved with reinforcement learning and this is the solution I will
  try to solve. For a human, walking is a simple task but we did learn after
  many trials, step by step, with the help of our hand at the beggining and then
  stand and walk.
  \paragraph{}
  To teach that to a 3D avatar, we need to define algorithm
  which allow not only the avatar to walk but to walk with a human gait to avoid
  too much movement for a single step, or avoid doing a false movement and fail
  down after three steps because of that false movement. At each step, without
  thinking about it, humans take care of the two/three steps we will do in the
  future. We have to design our model to be able to do that and optimally with
  the same gait as human, that we can consider as the optimal gait for our shape.
	
	\section{Datasets and Inputs}

  I will use OpenAI with the Gym python library\cite{1606.01540}, load the
  Roboschool environment (because the default Mujoco is not free) and use deep
  reinforcement learning to make the robot stands and walks.
  
  \subsection{Action space}
  The action space is a vector of 17 float values in the range [-1, 1]. Each
  value corresponds to the joints of the avatar by this order
  \href{https://github.com/openai/roboschool/blob/master/roboschool/mujoco_assets/humanoid_symmetric.xml}{XML}:
  \begin{multicols}{2}
  \begin{itemize}
  \item{abdomen\_y}
  \item{abdomen\_z}
  \item{abdomen\_x}
  \item{right\_hip\_x}
  \item{right\_hip\_z}
  \item{right\_hip\_y}
  \item{right\_knee}
  \item{left\_hip\_x}
  \item{left\_hip\_z}
  \item{left\_hip\_y}
  \item{left\_knee}
  \item{right\_shoulder1}
  \item{right\_shoulder2}
  \item{right\_elbow}
  \item{left\_shoulder1}
  \item{left\_shoulder2}
  \item{left\_elbow}
  \end{itemize}
\end{multicols}
  At each step, these values are applied to all the joints of the body by the code
\begin{verbatim}
for n,j in enumerate(self.ordered_joints):
    j.set_motor_torque( self.power*j.power_coef \
                         *float(np.clip(a[n], -1, +1)) )
\end{verbatim}

in the \verb?apply_action? function in the class which extends the
\verb?gym.Env? class (\verb?RoboschoolMujocoXmlEnv?) to set the torque value
into the respective motor.

\subsection{Observation space}
The state space (or observation space) is a vector of 44 float values in the
range [-5, 5] (Roboschool clip the vector with numpy before returning it in the
\verb?step? function). That vector is a concatenation of three subvectors:
\begin{itemize}
    \item{\textbf{more}: It is a vector of 8 values defined as follows:
        \begin{itemize}
            \item{The distance between the last position of the body and the current one.}
            \item{The sinus of the angle to the target.}
            \item{The cosinus of the angle to the target.}
            \item{The three next values is the X, Y and Z values of the matrix multiplication between
                \begin{itemize}
                   \item{\[\left(
 \begin{matrix}
  \cos(-yaw) & -\sin(-yaw) & 0 \\
  \sin(-yaw) & \cos(yaw) & 0 \\
  0 & 0 & 1
 \end{matrix}\right)
\]}
                   \item{The speed vector of the body.}
                \end{itemize}}
            \item{The roll value of the body}
            \item{The pitch value of the body}
        \end{itemize}}
     \item{\textbf{j}: This is the current relative position of the joint described earlier and their current speed. The position is in the even position, and the speed in the odds (34 values).}
     \item{\textbf{feet\_contact}: Boolean values, 0 or 1, for left and right feet, indicating if the respective feet is touching the ground or not.}
\end{itemize}

\subsection{Reward}

The reward is a sum of 5 computed values: \begin{itemize}
  \item{\textbf{alive}: -1 or +1 wether is on the ground or not}
  \item{\textbf{progress}: potential minus the old potential. The potential is defined by
    the speed multiplied by the distance to target point, to the negative.}
  \item{\textbf{electricity\_cost}: The amount of energy needed for the last action}
  \item{\textbf{joints\_at\_limit\_cost}: The amount of collision between joints of body
      during the last action}
  \item{\textbf{feet\_collsion\_cost}: The amount of feet collision taken during the last action}
  \end{itemize}
	
	\section{Solution Statement}

  In my research, I figured out two I can't decide which one would be the best
  to train my avatat to walk: A2C (Advantage Actor Critic) and Deep Q-Learning.
  For the sake of my learning, I am really interested in the A2C algorithm since
  it is the one who made great progress in Reinforcment Learning (I am thinking
  about AlphaGO\cite{silver2017mastering}). But I did understand that this
  algorithm will suit more on that problem since the walk of the robot is a
  continuous learning, and not an episodic, for which Deep Q-Learning is more
  suitable.
  
	\section{Benchmark Model}

  The random action model makes the avatar lying down on the ground convulsing
  because it doesn't know how to stand up and it is just moving its joints
  randomly.

  \begin{figure}[ht]
    \centering
    \includegraphics[width=.5\textwidth,height=.5\textheight,keepaspectratio]{Humanoid}
  \end{figure}

  In the above benchmark, we can see that Advantage Actor Critc algorithm is
  able to converge at around 100000 episodes. That tells me that I have to be
  patient on my training and have good metrics to tell if my model will converge
  or not.
  
	\section{Evaluation Metrics}
	
	\section{Project Design}

  \clearpage
  \bibliography{proposal}
  \bibliographystyle{plainnat}
  
  \printglossary
\end{document}
