\documentclass{article}
\usepackage[english]{babel}
\usepackage[square,comma,sort,numbers]{natbib}
\usepackage{glossaries}

\makeglossaries
\newglossaryentry{ZMP}{
  name=Zero Moment Point,
  description={It specifies the point with respect to which dynamic reaction force at the contact of the foot with the ground does not produce any moment in the horizontal direction, i.e. the point where the total of horizontal inertia and gravity forces equals 0 (zero).}
}

\title{\textbf{Capstone Proposal}\\Train a 3D avatar to walk}
\date{\today}
\author{Guitard Alan}

\begin{document}
	\maketitle	
	\section{Domain Background}

  For a human, the ability to walk is one of the most basic knowledge he learnt
  during his childhood. Since we are trying with AI to, when we can't do better,
  mimic the ability of human being, the idea of teaching a robot to walk comes
  naturally in our mind.\\
  One of the first attempt to do that was to develop an algorithm by trying to
  understand the physics formula of that movement. In 1993,
  \citet{1993-TrunkMotion} tried to make a little robot by trunk motion
  using the ZMP (\gls{ZMP}) and the three
  axis (pitch, yaw and roll)\cite{1993-TrunkMotion}. In 1999, \citeauthor{1999-KHR-2}
  added the OGM (Optimal Gradient Method) to that approach
  to "optimizes the horizontal motion of a trunk to reduce the
  deviation of the calculated ZMP from its
  reference." \cite{1999-KHR-2} and design the KHR-2 robot.
  The result was good but the gait of the robot wasnot very natural because
  it is very difficult to take all the factors in account for making
  the robot walk with a human gait.\\
  \cite{2013-TOG-MuscleBasedBipeds}
  
  
	\section{Problem Statement}
	
	\section{Datasets and Inputs}
	
	\section{Solution Statement}
	
	\section{Benchmark Model}
	
	\section{Evaluation Metrics}
	
	\section{Project Design}

  \clearpage
  \bibliography{proposal}
  \bibliographystyle{plainnat}
  
  \printglossary
\end{document}
